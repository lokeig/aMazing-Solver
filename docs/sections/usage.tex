When entering the website, you are presented with a grid that fills most of the screen. The size of the grid changes when you zoom in or out and refresh the page. On the left side, there is a green node and on the opposite side, on the right, there is a red node. The green node represents the starting position of the maze, while the red node marks the goal. The nodes can be moved by hovering the mouse cursor on top of them and holding down the left mouse button. They will follow the cursor until the mouse button is let go.

Clicking on a blank node on the grid turns it black, which means that it is a wall that cannot be walked through. Clicking on a black node turns it back to blank (white-colored) meaning it is walkable.

At the top of the screen, a dark blue header contains four clickable buttons in its center.

\begin{enumerate}
    \item </> - opens a modal window with a text editor for writing code. Will show the selected algorithm written in Amazing language. On the top right of the window an icon can be clicked opening a new tab to the projects GitHub page.
    \item Algorithm - opens a drop-down menu with all available maze solving algorithms. Clicking on one selects it to be used to solve mazes. 'Custom' uses the code written in the text editor.
    \item Generate Maze -  generates a maze that fills the grid.
    \item Clear Board - Returns the state of the grid to its original position by turning every node except the green and red blank.
\end{enumerate}

On the right side of the header, there is a button that says Visualize. Clicking on it starts an animation of the selected algorithm that attempts to navigate the current maze. First purple is drawn to show nodes that the algorithm has visited.  After visiting the end node, a yellow path is drawn from green to red, showing the solution the algorithm found.

Below the header, an explanation of all colors can be found.