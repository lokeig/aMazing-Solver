\subsection{Mazes}

\subsubsection{Interface}

The core types that different parts of the program interface with are mazes, paths, maze solvers and maze generators. A maze stores its start and end positions, the width and height of the maze and a two-dimensional array of cells describing the maze where each cell can be either a wall or empty. A path is an array of actions where each action can either be checking wether a cell at a position is a wall or to move one step in a cardinal direction. A maze solver is a function that takes a maze as input and returns a path that solves that maze. A maze generator takes a width and a height as input and returns a randomly generated maze as output.

\subsubsection{Path Verification}

The verify\_path function takes a path and a maze as parameters and returns wether that path successfully moves from the start to the end of the maze without ever moving into a wall cell.

\subsubsection{Solver Wrapper}

To aid the creation of maze solvers, a solver\_wrapper function has been created that takes a function as input and returns a maze solver. The wrapper defines variables to keep track of the current position in the maze and the path it has taken to get there. The input function is then passed five arguments: the end position of the maze, a function to get the current position in the maze, a function to check wether a position lies within the maze, a function to check wether a cell at a position is a wall and a function to move one step in a cardinal direction. These functions also update the current position and path accordingly.

\subsection{Maze Solving Algorithms}

\subsubsection{A*}

A* \cite{a-star-alg} is implemented as a maze solver using the solver wrapper. It will always find the shortest path and always halt.

\subsubsection{Depth First Search}

DFS \cite{pkd} is implemented as a maze solver using the solver wrapper. It will not always find the shortest path but will always halt.

\subsubsection{Dijkstra's Algorithm}

Dijkstra's algorithm \cite{dijkstras-alg} is implemented as a maze solver using the solver wrapper. It will always find the shortest path and always halt.

\subsubsection{The Maze Routing algorithm}

The maze routing algorithm \cite{maze-routing-alg} is implemented as a maze solver using the solver wrapper. It will not always find the shortest path and may not halt when the maze is not solvable.

\subsubsection{aMazing Language Implementations}

In addition to typescript, each maze solving algorithm used is also implemented as strings executable by the aMazing Language in maze mode.

\subsection{Maze Generation}
\subsubsection{Recursive Division}
Recursive division is a maze generation algorithm that works by recursively dividing the board into smaller sections, adding horizontal and vertical walls with passage openings, and continues dividing until the sections reach a minimum size \cite{recursive-division-alg}.

\subsection{Component Overview}
React is a library that lets you build user interfaces with components. In a React app, a component is a reusable piece of UI defined as a function that returns JSX markup \cite{react-component}. This component overview provides a detailed explanation of the components.

\subsubsection{App}
The \texttt{App} component is the entry point to the React app.

\subsubsection{Context}
The \texttt{GridContext} and \texttt{EditorContext} components use React’s Context API to create contexts that allow components to share state and stateful logic without the need to explicitly pass props \cite{react-context}.

\paragraph{GridContext and EditorContext} \

The \texttt{Editor} and \texttt{Board} components share information with the \texttt{Header} component. Contexts establish a single source of truth from which the components can derive their data. GridContext defines a \texttt{GridProvider} component that wraps the components that need access to the shared information. In \texttt{App}, the \texttt{GridProvider} wraps both the \texttt{Header} and \texttt{Board} components, allowing them to share the \texttt{grid} and \texttt{disabled} states, as well as the stateful logic, which includes the functions \texttt{setGrid} and \texttt{setDisabled} to manage them \cite{react-hooks}.

\begin{verbatim}
    const [grid, setGrid] = useState<Grid>(makeGrid(0, 0));
    const [disabled, setDisabled] = useState<boolean>(false);
\end{verbatim}

Similarly, \texttt{EditorContext} defines an \texttt{EditorProvider} component that shares information about the code in the editor, logs and the stateful logic.

To access shared information, we define custom Hooks called \texttt{useGrid} and \texttt{useEditor} that can be imported into other components.

\begin{verbatim}
    const { grid, setGrid, disabled, setDisabled } = useGrid();
\end{verbatim}

\subsubsection{Board}
The \texttt{Board} component is the interactive grid where the maze-solving algorithms are visualized.

\paragraph{Data Types} \

\texttt{Board} uses two important types:
\begin{itemize}
    \item \texttt{Node}: Represents a single cell in the grid with properties about its row and column indices, and optional properties for the start node, end node, and walls (isStart, isEnd, isWall).
    \item \texttt{Grid}: Represents the entire grid as a 2D array of nodes, references to the start and end nodes, and the number of rows and columns.
\end{itemize}

\paragraph{Integration with GridContext} \

The component uses \texttt{useGrid}, which provides the grid state and a function \texttt{setGrid} to manage the grid state and trigger a new render when the state changes.

The \texttt{disabled} state in the \texttt{Board} component controls if user interactions are allowed. When the \texttt{disabled} state is set to true, the board receives the "disabled" class, which sets pointer events to none in CSS, preventing interactions with the board are prevented.

\begin{verbatim}
    <div
        ref={boardRef}
        className={clsx({"disabled": disabled}, "board")}
        onMouseUp={handleMouseUp}
    >
\end{verbatim}

\paragraph{Grid Initialization} \

When the \texttt{Board} is mounted, a \texttt{useEffect} Hook \cite{react-hooks} gets dimensions using \texttt{boardRef} and calculates the number of rows and columns based on the size of a 24-pixel cell. Refs, created with the \texttt{useRef} Hook, remember information without triggering new renders and give access to DOM elements in React \cite{react-hooks}. The grid is created using \texttt{makeGrid}, and the initial grid is updated with the new grid. The effect will run only once because \texttt{setGrid} is stable and does not change between renders, which causes the effect to run only once on mount (when the component first renders). The grid is initially defined in \texttt{GridContext}, but because the \texttt{Board} component is not rendered at this point and the dimensions are based on the layout defined by CSS, the dimensions cannot be retrieved until after the component has been mounted. This is why \texttt{useEffect} is used here.

\paragraph{Interacting with the Grid} \

The \texttt{Board} component enables user interactions through mouse event handlers.

\begin{itemize}
    \item \texttt{handleMouseDown}: This function is called when a mouse button is pressed on a cell and checks if the clicked cell is a start node, an end node, or a regular node. Then, it assigns an action to \texttt{nodeRef} or \texttt{wallRef}.
    \item \texttt{handleMouseEnter}: This function is called when the mouse moves over a cell while the mouse button is pressed down. It uses helper functions to move the start node or end node and modifies walls.
    \item \texttt{handleMouseUp}: When a mouse button is released, it resets the references \texttt{nodeRef} and \texttt{wallRef} to null and updates the grid state to sync changes made during interactions.
\end{itemize}

\paragraph{Rendering} \

The grid is rendered as an HTML table, where each cell is a node in the grid and has a unique ID that can be retrieved using the \texttt{getNodeID} function. Using IDs helps when manipulating cells in the DOM.

The update logic decouples visual feedback from the underlying grid state for performance reasons. When the user interacts with the board, the component uses helper functions to manipulate the DOM by adding or removing CSS classes on the cells to reflect the changes visually. When an interaction is completed, \texttt{updateGrid} is called to update the grid state.

\paragraph{Helper Functions} \

\begin{itemize}
    \item \texttt{editWall}: Modifies the class attributes of a specific cell. If the cell is not a start or end node, it changes the \texttt{isWall} property of the node based on whether the current action in \texttt{wallRef} is to add or remove a wall. This function adds or removes the corresponding CSS class to visually update the cell.
    \item \texttt{moveNode}: Repositions of the start and end nodes. First, it ensures that the target cell is not a wall and is not occupied by the opposite node. Then, it moves the node by removing the relevant CSS class from the old cell and adding it to the new one, while also updating the \texttt{isStart} or \texttt{isEnd} properties of the nodes.
    \item \texttt{styles}: Uses the clsx library to conditionally return a string of CSS classes for a given node. This helps in applying different styles for the cells (e.g. walls, start, and end nodes) based on the properties of the node.
\end{itemize}

\paragraph{Visualizing Algorithms} \

Algorithms are visualized on the board with an asynchronous function called \texttt{visualize}. Because the \texttt{Grid} type is not directly compatible with the \texttt{Maze} type, which is used by the maze-solving algorithms, a helper function \texttt{gridToMaze} is used to convert the \texttt{Grid} to a \texttt{Maze}. The reason for the asynchronous approach is to see the progression of the algorithm searching and drawing the path, as the DOM updates would otherwise happen instantly. Each step of the visualization is delayed by 10 milliseconds using \texttt{setTimeout}. The Promise API provides a \texttt{resolve} function, which is used to indicate that an asynchronous operation (the delay) is complete and allow the next step.

\subsection{aMazing Language Interpreter}

\subsubsection{Tokenizer}

The first step of interpreting a string is to convert it to an array of tokens. A token consists of its type, the text it represents, the line number it started on and the column number it started on.

The input string is read one character at a time and that character is added to a string containing the current token. When whitespace is encountered or the current token string would become invalid, the current token string is matched against keyword, symbols, integer literal rules and variable naming rules. It it matches any of them, a new token of that type is pushed to the resulting array and the current token string is reset to begin accumulating the next token.

\subsubsection{Parser}

Second is to convert the token array of tokens into intermediate representation. The intermediate representation is a tree-like structure where each node contains the type of expression or statement it represents as well as what other values and nodes it consists of.

The input array is treated as a stack from which tokens are consumed as subexpressions are parsed. The main parse function will look at the next token and depending on what it is, it will call different helper functions capable of parsing that specific type of statement. Those functions will in turn call different helpers to parse the expressions they consist of which will call helpers to parse their subexpressions. Once all subexpressions are parsed they can be combined into the full expression node and returned to be combined into a full statement node and so on.

\subsubsection{Evaluator}

The evaluator will use the tokenizer and parser to get the inputs intermediate representation. It uses environment frame to store a table for looking up the values associated with variable names as well as what frame to fall back on if a name could not be found.

The predefined values are defined in a new environment frame before starting the evaluation of the intermediate representation in a new frame with that as fallback. Similarly to the parser, when evaluating statements, helper functions are called to parse the expressions that make them up which in turn call helper functions that evaluate their subexpressions and so on. Eventually, a function is returned that will evaluate the main function of the input using this same method.

\subsubsection{Maze Solving Evaluator}

To evaluate in maze mode, the solver wrapper is used. The regular evaluator is used but the functions passed by the solver wrapper are exposed to the program as additional predefined values. Additionally, the main function is called with the end position's coordinates as arguments. A tuple of a maze solver and an array of strings is returned. The array of strings represents the programs standard output and will be filled with the strings printed during execution after the maze solver is called.
