\subsection{Tokens}

\subsubsection{Comments}

The first \verb|#| character in a line and all characters after it until the next new line or end of input is ignored.

\paragraph{Example}

\begin{verbatim}
    var x = 2; # this is a declaration
    #abc # var var var ?
\end{verbatim}

\subsubsection{Naming Rules}

A variable name is case sensitive and must start with a letter a-z or A-Z or an underscore, and every following character must be a letter a-z or A-Z, an underscore or a digit 0-9. A variable cannot share a name with any of the reserved keywords.

\paragraph{Example}

\begin{verbatim}
    x
    _x
    x1
    _123
\end{verbatim}

\subsubsection{Integers}

An integer literal is case insensitive and must start with a digit 0-9 and every following character must be a letter a-z or A-Z, an underscore or a digit 0-9. Underscores are ignored for purposes of determining its value.

If it begins with \verb|0b|, it is a binary integer and can only contain binary digits. At least one non-underscore character must follow.

If it begins with \verb|0x|, it is a hexadecimal integer and can only contain hexadecimal digits. At least one non-underscore character must follow.

Otherwise, it is a decimal integer and can only contain decimal digits.

\paragraph{Example}

\begin{verbatim}
    1234     → 1234
    1_234    → 1234
    0x1f     → 31
    0x_1__f_ → 31
    0b01010  → 10
    0b1_010  → 10
\end{verbatim}

\subsubsection{Reserved Keywords}

\begin{itemize}
    \item \verb|var|
    \item \verb|fn|
    \item \verb|if|
    \item \verb|else|
    \item \verb|while|
    \item \verb|for|
    \item \verb|return|
    \item \verb|continue|
    \item \verb|break|
\end{itemize}

\subsubsection{Operators}

\paragraph{Unary Postfix Operators}

All unary postfix operators have the same precedence which is higher than that of all unary prefix operators. All unary postfix operators are left-to-right associative.

\begin{table}[H]
    \begin{tabular}{ l l }
        \verb|()| & Function call      \\
        \verb|[]| & Array subscripting \\
    \end{tabular}
\end{table}

\paragraph{Unary Prefix Operators}

All unary prefix operators have the same precedence, which is higher than that of all binary operators. All unary prefix operators are right-to-left associative.

\begin{table}[H]
    \begin{tabular}{ l l }
        \verb|!| & Logical NOT \\
        \verb|+| & Unary plus  \\
        \verb|-| & Unary minus
    \end{tabular}
\end{table}

\paragraph{Binary Operators}

Binary operators are listed top to bottom in descending precedence. Operators on the same line have the same precedence. All binary operators are left-to-right associative.

\begin{table}[H]
    \begin{tabular}{ l l }
        \verb|*| \verb|/| \verb|%|            & Multiplication, division and modulo \\
        \verb|+| \verb|-|                     & Addition and subtraction            \\
        \verb|<| \verb|<=| \verb|>| \verb|>=| & The common relational operators     \\
        \verb|==| \verb|!=|                   & Equality and inequality             \\
        \verb|&&|                             & Logical AND                         \\
        \verb!||!                             & Logical OR
    \end{tabular}
\end{table}

\paragraph{Example}

\begin{verbatim}
    x[y]()      → (x[y])()
    !-x         → !(-x)

    x + y * z   → x + (y * z)
    x - y + z   → (x - y) + z
    x && y || z → (x && y) || z

    -x[y]       → -(x[y])
    -x + -y     → (-x) + (-y)
    x() + y[z]  → (x()) + (y[z])
\end{verbatim}

\subsubsection{Other Symbols}

All symbols (including operators) are parsed greedily, not ending a symbol before it would become invalid or encounter whitespace.

\begin{table}[H]
    \begin{tabular}{ l l }
        \verb|(| \verb|)| & Overriding precedence      \\
        \verb|{| \verb|}| & Grouping statements        \\
        \verb|[| \verb|]| & Creating arrays            \\
        \verb|=|          & Declaration and assignment \\
        \verb|,|          & Separating elements        \\
        \verb|;|          & End statement
    \end{tabular}
\end{table}

\paragraph{Example}

\begin{verbatim}
    ===  → == =
    =!=  → = !=
    |||| → || ||
\end{verbatim}

\subsection{Syntax}

\subsubsection{Expressions}

Expressions must be of one of the following forms.

\begin{table}[H]
    \begin{tabular}{ l l }
        \textit{name}                                                                                               & Variable name         \\
        \textit{integer}                                                                                            & Integer literal       \\
        \verb|(| \textit{expression} \verb|)|                                                                       & Overriding precedence \\
        \verb|[| \textit{expression} \verb|,| \ldots \verb|]|                                                       & Array literal         \\
        \verb|fn| \verb|(| \textit{expression} \verb|,| \ldots \verb|)| \verb|{| \textit{statement} \ldots \verb|}| & Function literal      \\
        \textit{expression} \verb|(| \textit{expression} \verb|,| \ldots \verb|)|                                   & Function call         \\
        \textit{expression} \verb|[| \textit{expression} \verb|]|                                                   & Array subscripting    \\
        \textit{unary-prefix-operator} \textit{expression}                                                          & Unary prefix operator \\
        \textit{expression} \textit{binary-operator} \textit{expression}                                            & Binary operator
    \end{tabular}
\end{table}

\paragraph{Example}

\begin{verbatim}
    [1, 2, 3] + [x, y, z]
    fn () {}
    (fn (x, y) { return x + y; })(1, 1)
    -x[y]
    2 * (1 + 1)
\end{verbatim}

\subsubsection{Statements}

Statements must be of one of the following forms.

\begin{table}[H]
    \begin{tabular}{ m{225pt} l }
        \verb|;|                                                                                                                                                                                                        & No operation         \\
        \textit{expression} \verb|;|                                                                                                                                                                                    & Expression           \\
        \verb|var| \textit{name} \verb|=| \textit{expression} \verb|;|                                                                                                                                                  & Variable declaration \\
        \textit{expression} \verb|=| \textit{expression} \verb|;|                                                                                                                                                       & Variable assignment  \\
        \verb|if| \verb|(| \textit{expression} \verb|)| \textit{statement}                                                                                                                                              & If conditional       \\
        \verb|if| \verb|(| \textit{expression} \verb|)| \textit{statement} \verb|else| \textit{statement}                                                                                                               & If-else conditional  \\
        \verb|while| \verb|(| \textit{expression} \verb|)| \textit{statement}                                                                                                                                           & While loop           \\
        \verb|for| \verb|(| \textit{expression/assignment/declaration} \verb|;| \newline \hspace*{29pt} \textit{expression} \verb|;| \textit{expression/assignment} \verb|)| \newline \hspace*{29pt} \textit{statement} & For loop             \\
        \verb|return| \verb|;|                                                                                                                                                                                          & Return               \\
        \verb|return| \textit{expression} \verb|;|                                                                                                                                                                      & Return value         \\
        \verb|continue| \verb|;|                                                                                                                                                                                        & Continue             \\
        \verb|break| \verb|;|                                                                                                                                                                                           & Break                \\
        \verb|{| \textit{statement} \ldots \verb|}|                                                                                                                                                                     & Code block           \\
    \end{tabular}
\end{table}

\paragraph{Example}

\begin{verbatim}
    var x = 1;
    x = 2;
    if (x) return x; else if (y) continue; else break;
    while (f(x));
    for (var i = 0; i < 10; i = i + 1) {}
    { var x = 1; x = 2; }
\end{verbatim}

\subsection{Types}

There are three types: integer, array and function. Variables, elements, parameters and return values do not have types associated with them and can store, accept and return values of different types at different times. No two values of different types are equal.

\paragraph{Integer}

An integer stores a whole number. Two integers are equal if they represent the same number.

\paragraph{Array}

An array stores a reference to an ordered list of elements that can be of different types. Elements can be modified, added or removed from the list and all arrays storing the same reference will observe the effect. Two arrays are equal if they store the same reference.

\paragraph{Function}

A function stores a reference to an object that can be called with a list of ordered values that can be of different types as its arguments to return a value of any type. Two functions are equal if they store the same reference.

\subsection{Environment}

The first frame created, which points to nothing, contains all predefined values. This is what the program frame, where the user-written code runs, points to. After executing, the program frame must contain a variable named \verb|main| that stores a function. The number and types of the arguments this function is called with will depend on the use case.

\paragraph{Creation}

When a statement inside a conditional, loop or code block is executed, it will do so inside a newly created frame pointing to the frame the condition, loop or code block was executed inside. Additionally, a function object will store what frame it was created in and when called, the body will be executed in a new frame pointing to the one stored in the function object. The function body is executed in the same frame the parameters were declared in.

\paragraph{Usage}

When reading or assigning to a variable, the name will be searched for within the current frame, and if it has not been declared, it will recursively search the frame it points to. If it has not been declared and does not point to anything, an error is thrown. When declaring a variable, if the name has not been declared in the current frame, it will be, otherwise an error is thrown. Note that if a variable is read or assigned to but would be declared later in the same scope, the reading or assigning will still search the frame pointed to.

\paragraph{Example}

\begin{verbatim}
    var x = 1;
    {
        print(x); # 1
        x = 3;
        var x = 2;
        print(x); # 2
    }
    print(x) # 3
\end{verbatim}

\subsection{Semantics}

\subsubsection{Truth Value}

An integer is truthy if it is not 0. An array is truthy if its length is not 0. A function is always truthy.

\subsubsection{Operators}

\paragraph{Function call \quad \texttt{f(x)}} \

Takes a function as its primary operand and optionally a list of values between the parentheses as arguments to the function. Returns the value returned by its body or 0 if no return value was specified.

\paragraph{Array subscripting \quad \texttt{a[i]}} \

Takes an array as its primary operand and an integer between the brackets to use as index. The index must be between 0 (inclusive) and the length of the array (exclusive) where 0 corresponds to the first element in the array. Returns the value stored at that index, which can be assigned to.

\paragraph{Logical NOT \quad \texttt{!x}} \

Takes a value as an operand and returns 0 if it is truthy and 1 otherwise.

\paragraph{Unary plus \quad \texttt{+i}} \

Takes an integer and returns the same integer.

\paragraph{Unary minus \quad \texttt{-i}} \

Takes an integer and returns its additive inverse.

\paragraph{Multiplication \quad \texttt{i*i}} \

Takes two integers and returns their product.

\paragraph{Division \quad \texttt{i/i}} \

Takes two integers and returns the floor of their quotient. If the right operand is 0 an exception is thrown.

\paragraph{Modulo \quad \texttt{i\%i}} \

Takes two integers and returns the integer $ a - b \left\lfloor \frac{a}{b} \right\rfloor $ where a is the left operand and b the right operand. If the right operand is 0 an exception is thrown.

\paragraph{Addition / Concatenation \quad \texttt{x+x}} \

Takes two integers or two arrays. If they are integers, it returns their sum. If they are arrays, it returns a new array containing all the elements in the left operand followed by all the elements in the right operand.

\paragraph{Subtraction \quad \texttt{i-i}} \

Takes two integers and returns their difference.

\paragraph{Less than \quad \texttt{i<i}} \

Takes two integers and returns 1 if the left operand is less than the right operand, or 0 otherwise.

\paragraph{Less than or equal to \quad \texttt{i<=i}} \

Takes two integers and returns 1 if the left operand is less than or equal to the right operand, or 0 otherwise.

\paragraph{Greater than \quad \texttt{i>i}} \

Takes two integers and returns 1 if the left operand is greater than the right operand, or 0 otherwise.

\paragraph{Greater than or equal to \quad \texttt{i>=i}} \

Takes two integers and returns 1 if the left operand is greater than or equal to the right operand, or 0 otherwise.

\paragraph{Equality \quad \texttt{x==x}} \

Takes two values and returns 1 if they are equal, or 0 otherwise.

\paragraph{Inequality \quad \texttt{x!=x}} \

Takes two values and returns 0 if they are equal, or 1 otherwise.

\paragraph{Logical AND \quad \texttt{x\&\&x}} \

Evaluates the left operand and returns it if it is not truthy. Otherwise, it evaluates the right operand and returns it.

\paragraph{Logical NOT \quad \texttt{x||x}} \

Evaluates the left operand and returns it if it is truthy. Otherwise, it evaluates the right operand and returns it.

\subsubsection{Statements}

\paragraph{No operation \quad \texttt{;}} \

Does nothing.

\paragraph{Expression \quad \texttt{x;}} \

Evaluates the expression, discarding its value.

\paragraph{Variable declaration \quad \texttt{var n = x;} } \

Declares a variable with the name following the var keyword in the current frame and assigns it the value on the right-hand side of the equal sign.

\paragraph{Variable assignment \quad \texttt{x = x;} } \

The expression on right-hand side of the equals sign is evaluated before the expression of the left-hand sign.

If the left-hand side of the equals sign is a variable name, that variable is assigned the value on the right-hand side of the equals sign.

If the left-hand sign is an array subscripting expression, the element at that index of the array is assigned the value on the right-hand side of the equal sign.

Otherwise an error is thrown.

\paragraph{If conditional \quad \texttt{if (x) s} } \

If the value inside the parentheses is truthy, the statement following them is executed in a new frame. Any returns, continues and breaks are passed through.

\paragraph{If-else conditional \quad \texttt{if (x) s else s} } \

If the value inside the parentheses is truthy, the statement following them is executed in a new frame. Otherwise, the statement following the else keyword is executed in a new frame. Any returns, continues and breaks are passed through.

\paragraph{While loop \quad \texttt{while (x) s}} \

Evaluates the expression inside the parentheses in a new frame. If it is truthy, the statement following them is evaluated. If it is a return, the return is carried through. If it is a break, the loop stops. Otherwise, this is repeated.

\paragraph{For loop \quad \texttt{for (x; x; x) s}} \

Evaluates the first expression inside the parentheses in a new frame. Evaluates the second expression inside the parentheses. If that second expression is truthy, the statement following the parentheses is evaluated in a new frame. If it is a return, the return is carried through. If it is a break, the loop stops. Otherwise, the third expression inside the parentheses is evaluated and everything except the evaluation of the first expression is repeated.

\paragraph{Return \quad \texttt{return;}} \

Skips all following statements when inside a code block and returns 0 when inside a function.

\paragraph{Return value \quad \texttt{return x;}} \

Skips all following statements when inside a code block and returns the value following the return keyword when inside a function.

\paragraph{Continue \quad \texttt{continue;}} \

Skips all following statements when inside a code block.

\paragraph{Break \quad \texttt{break;}} \

Skips all following statements when inside a code block and stops the loop when inside one.

\paragraph{Code block \quad \texttt{\{s\ldots\}}} \

Evaluates all statements inside the curly brackets in a new frame. If a statement is a return, continue, or break, the statements following it are not evaluated, and the return, continue, or break is passed through.

\subsubsection{Predefined Values}

\paragraph{\texttt{true}} \

The integer 1.

\paragraph{\texttt{false}} \

The integer 0.

\paragraph{\texttt{panic}} \

A function that takes no arguments and throws an error.

\paragraph{\texttt{print}} \

A function that takes any number of arguments of any type and prints them separated by spaces. When printing arrays, a maximum depth of 5 nested arrays are printed.

\paragraph{\texttt{len}} \

A function that takes an array as its only argument and returns its length as an integer.

\paragraph{\texttt{push}} \

A function that takes an array as its first argument and a value of any type as its second argument and then adds the second argument to the end of the array, increasing its length by one. Returns 0.

\paragraph{\texttt{pop}} \

A function that takes an array as its only argument. If the array is empty, an error is thrown. Otherwise, the last element of the array is removed, reducing its length by one. Returns the removed element.

\paragraph{\texttt{is\_int}} \

A function that takes a value as its only argument and returns 1 if it is an integer, or 0 otherwise.

\paragraph{\texttt{is\_arr}} \

A function that takes a value as its only argument and returns 1 if it is an array, or 0 otherwise.

\paragraph{\texttt{is\_fun}} \

A function that takes a value as its only argument and returns 1 if it is a function, or 0 otherwise.

\paragraph{Example}

\begin{verbatim}
    print(true, false, len([3, 2]));      # "1 0 2"
    var x = [1, 2]; push(x, 3); print(x); # "[1, 2, 3]"
    var x = [1, 2]; print(pop(x), x);     # "3, [1, 2]"
    print(is_int(true));                  # "1"
    print(is_arr([3, 2][0]));             # "0"
    print(is_fun(is_fun));                # "1"
    print([[[[[[[[[[]]]]]]]]]]);          # "[[[[[...]]]]]"
\end{verbatim}

\subsection{Maze Mode}

Maze mode is meant to aid in creating maze solving algorithms using the language and is what is used by aMazing Solver.

\subsubsection{Mazes}

A maze is a grid of cells with a current position and a goal position. Each cell in the grid can either be a wall or not a wall, and a valid maze solving algorithm should never occupy the same space as a wall. The top left cell has x-coordinate 0 and y-coordinate 0, and the x and y coordinates of cells increase as you move right and down through the maze, respectively.

\subsubsection{The main Function}

The main function will be called with two integers, the first being the x-coordinate of the goal, and the second being the y-coordinate of the goal. If the main function returns when the current position is the same as the goal position and the current position has never overlapped with a wall or exceeded the boundaries of the maze, it has successfully solved the maze. The return value is ignored.

\subsubsection{Additional Predefined Values}

\paragraph{\texttt{right}} \

The integer 0.

\paragraph{\texttt{up}} \

The integer 1.

\paragraph{\texttt{left}} \

The integer 2.

\paragraph{\texttt{down}} \

The integer 3.

\paragraph{\texttt{get\_x}} \

A function that takes no arguments and returns the current x-coordinate within the maze as an integer.

\paragraph{\texttt{get\_y}} \

A function that takes no arguments and returns the current y-coordinate within the maze as an integer.

\paragraph{\texttt{in\_bound}} \

A function that takes two integer arguments and returns 1 if the position with x value equal to the first argument and y value equal to the second argument is within the bounds of the maze. Otherwise, it returns 0.

\paragraph{\texttt{is\_wall}} \

A function that takes two integer arguments and returns 1 if the position with x value equal to the first argument and y value equal to the second argument is within the bounds of the maze or if that position is occupied by a wall. Otherwise, it returns 0.

\paragraph{\texttt{move}} \

A function that takes one integer value. If the value is 0, the current x-coordinate is incremented by one. If the value is 1, the current y-coordinate is decremented by one. If the value is 2, the current x-coordinate is decremented by one. If the value is 3, the current y-coordinate is incremented by one. Otherwise, an error is thrown.

\paragraph{Example}

\begin{verbatim}
    # Moves down if there is no wall below.
    if (!is_wall(get_x(), get_y() + 1)) {
        move(down);
    }
\end{verbatim}
